\documentclass[aspectratio=169]{beamer}

\usepackage[utf8]{inputenc}
\usepackage{minted,tcolorbox,ulem,listings}
\usepackage{hyperref}

\title{Onion Architecture}
\author{Ethan Kent}
\institute{Spoonflower}
\date{\today}

\begin{document}

\frame{\titlepage}

\begin{frame} % [fragile]
  \frametitle{Dependency Inversion Principle}
  Per Wikipedia:

  \begin{enumerate}
    \item High-level modules should not import anything from low-level modules. Both should depend on abstractions (e.g., interfaces).
    \item Abstractions should not depend on details. Details (concrete implementations) should depend on abstractions.
  \end{enumerate}

\end{frame}

\begin{frame}
  \frametitle{Dependency Inversion Principle, continued}

  \begin{itemize}
    \item Don't refer to volatile concrete classes. Refer to abstract
          interfaces instead.  This rule applies in all languages, whether
          statically or dynamically typed.~.~.~.
    \item Don't derive from volatile concrete classes. This is a corollary to
          the previous rule, but it bears special mention.~.~.~.
    \item Never mention the name of anything concrete and volatile. This is
          really just a restatement of the principle itself.
  \end{itemize}
  \vspace{1em}

  Robert C. Martin, \textit{Clean Architecture} 89 (2018).
\end{frame}

\begin{frame}
  \frametitle{Dependency Inversion Principle, continued}

  This is all pretty abstract. What are you actually saying?
\end{frame}

\begin{frame}
  \frametitle{Dependency Inversion Principle, continued}

  Let's say I have two concepts:

  \begin{itemize}
    \item A \texttt{user}, and
    \item an Express server with a \texttt{POST} endpoint called
          \texttt{updateUser}, which takes in a \textsc{json} payload and
          updates the PostgreSQL database to the new user; the server emits
          traces and metrics via Open Telemetry, and checks a bearer token to
          ensure the \texttt{admin} role is set properly.
  \end{itemize}
  \vspace{1em}
  Which is more stable, less likely to change, closest to our business domain,
  etc.?
\end{frame}

\begin{frame}[fragile]
  \frametitle{Dependency Inversion Principle, continued}
  Should our \texttt{User} type include this data?
  \vspace{1em}
  \begin{minted}{typescript}
interface User {
  givenName: string;
  middleName?: string;
  familyName: string;
  sqlId: number;
  roleFromToken: string;
  oTelCorrelationId: string;
  isAuthorizedToUpdate: boolean;
}
  \end{minted}
\end{frame}


\begin{frame}
  \frametitle{Dependency Inversion Principle, continued}
  \begin{quote}
    High-level modules should not import anything from low-level modules. Both
    should depend on abstractions (e.g., interfaces).
    \vspace{1em}
    \\
    Abstractions should not depend on details. Details (concrete
    implementations) should depend on abstractions.
  \end{quote}
\end{frame}

\begin{frame}
  \frametitle{Dependency Inversion Principle, continued}
  Why?
  \vspace{1em}
  \\
  \begin{quote}
    Depend in the direction of stability. \\

    Designs cannot be completely static. Some volatility is necessary if the
    design is to be maintained.~.~.~. Some~.~.~. components are designed to be
    volatile.  We expect them to change. \\

    Any component that we expect to be volatile should not be depended on by a
    component that is difficult to change.  Otherwise, the volatile component
    will also be difficult to change.
  \end{quote}\\
  \vspace{1em}
  Martin, \textit{supra}, at 120.
\end{frame}

\begin{frame}
  \frametitle{Dependency Inversion Principle, continued}
  Let's apply this thinking to our \texttt{User} type.

  \begin{itemize}
    \item Are \texttt{User}s supposed to be volatile?
    \item Will \texttt{User}s stop having names?
    \item Will our application always use SQL?
    \item Would a business stakeholder recognize an Open Telemetry correlation
          ID as part of the concept of a \textsc{User} in the ubiquitous
          language of the business domain?
  \end{itemize}
\end{frame}

\begin{frame}
  \frametitle{Dependency Inversion Principle, continued}
  In what way are we violating the Dependency Inversion Principle?
  \begin{itemize}
    \item A \texttt{User} is not supposed to be volatile, so it is the kind of
          thing that belongs in a low-level module.
    \item SQL, Open Telemetry, Express, Bearer Tokens, etc., are volatile,
          and so belong in high-level modules.
    \item Our \texttt{User}, in a low-level module, depends here on numerous
          high-level modules.
  \end{itemize}
\end{frame}

\begin{frame}
  \frametitle{Dependency Inversion Principle, continued}
  Okay, this makes sense, but this whole ``rely on abstraction not concretions''
  thing is a bunch of mumbo-jumbo.
\end{frame}

\begin{frame}
  \frametitle{Dependency Inversion Principle, continued}
  Fair enough.

  \vspace{1em}

  Sometimes it will be necessary for lower-level modules to interact.

  \vspace{1em}

  For example, it may be stable, non-volatile, fundamental to the business
  domain, etc.\ that---

  \begin{itemize}
    \item A \textsc{User} is authorized to do some things but not
          others.
    \item A \textsc{User} will be stored somewhere.
    \item The performance of the system will be monitored.
  \end{itemize}
\end{frame}

\begin{frame}[fragile]
  \frametitle{Dependency Inversion Principle, continued}
  We're going to be very careful to have a \texttt{User} type that is abstract
  and non-volatile, more or less in the very deepest, most core part of the
  application:

  \vspace{1em}

  \begin{minted}{typescript}
interface User {
  givenName: string;
  middleName?: string;
  familyName: string;
}
  \end{minted}

  \vspace{1em}

  Sure the \textsc{User} concept will be depended on by other, less general
  things, but the opposite shouldn't happen.
\end{frame}

\begin{frame}[fragile]
  \frametitle{Dependency Inversion Principle, continued}
  But now we have a conflict:

  \begin{itemize}
    \item The \emph{idea} of \textsc{Logging} is pretty abstract, but the
          \emph{implementation} via, say, \textsc{Pino} or \textsc{Winston} is
          concrete.
    \item The \emph{idea} of \textsc{Authorization} is pretty abstract, but the
          \emph{implementation} via, say, an \textsc{OAuth2}-compatible Bearer
          Token is concrete.
    \item \emph{Within} our \emph{abstract} notion of \textsc{Authorization}, we
          may want to refer to the \emph{abstract} notion of \textsc{Logging}, or
    \item Within our \emph{concrete} implementation of \textsc{Authorization} via an
          \textsc{OAuth2}-compatible Bearer Token, we might want to refer to the
          \emph{abstract} notion of \textsc{Logging} in order to avoid unnecessary
          coupling between different concerns.
  \end{itemize}
\end{frame}

\begin{frame}
  \frametitle{Dependency Inversion Principle, continued}
  In short, we sometimes want something more abstract and stable to have the
  notion of some other concept that is more volatile. Perhaps our abstract
  concept of a \textsc{Logger} needs to depend on the details of
  an \textsc{ID} that comes from the \emph{concrete} implementation of the
  abstract idea of \textsc{Persistence}.
\end{frame}

\begin{frame}
  \frametitle{Dependency Inversion Principle, continued}
  Let's pause for a second. Are you starting to get a mental image of layers
  here? And develop an instinct about which way things inside those layers
  should or should not know about one another?
\end{frame}

\begin{frame}
  \frametitle{Dependency Inversion Principle, continued}
  Back to the Dependency-Inversion Principle. How can a more abstract thing talk
  to a more concrete thing?

  \vspace{1em}

  \begin{quote}
    Abstractions should not depend on details. Details (concrete implementations) should depend on abstractions.
  \end{quote}
\end{frame}


\begin{frame}[fragile]
  \frametitle{Dependency Inversion Principle, continued}
  Define some abstractions:

  \vspace{1em}

  \begin{minted}{typescript}
interface AuthorizationService<T> {
  authenticate: (credential: T) => Promise<boolean>;
}

interface LoggingService {
  log: (message: string) => void;
}
\end{minted}
\end{frame}

\begin{frame}[fragile]
  \frametitle{Dependency Inversion Principle, continued}
  Define some concretions:

  \vspace{1em}

  \begin{minted}{typescript}
import jwt, { type Jwt } from "jsonwebtoken";
import jwksClient from "jwks-rsa";

const client = jwksClient({ jwksUri: process.env.JWKS_URI });

const getKey = (header, callback) => {
  client.getSigningKey(header.kid, (err, key) => {
    callback(null, key.publicKey || key.rsaPublicKey);
  });
}

const jwtAuth: AuthorizationService<Jwt> = {
jwt.verify(token, getKey, options, function(err, decoded) {
  console.log(decoded.foo) // bar
});
};
  \end{minted}
\end{frame}

\begin{frame}
  \frametitle{Dependency Inversion Principle, continued}
  Is this the same thing as Dependency \textit{Injection}?
\end{frame}
\end{document}
